%--------------------题目&摘要页----------------------
\newpage
\pagenumbering{arabic} %设置阿拉伯数字页码
\setcounter{page}{1} %正文为第一页

    \begin{center}
        \zihao{-2}  \bfseries \xinwei \shadowtext{中国研究生创新实践系列大赛}
    
        \zihao{2} \shadowtext{\textbf{ \xinwei “华为杯”第二十届中国研究生}}
    
        \zihao{2} \shadowtext{\textbf{\xinwei 数学建模竞赛}}
        \end{center}

\vspace{1em}
\begin{tabular}{l p{0.8\textwidth}<{\centering}}
    \centering
    \zihao{4} 题\quad 目\quad & \zihao{3} \heiti          \\ \cline{2-2}
\end{tabular}

\begin{center} \zihao{-2} \lishu 摘 \qquad 要:
\end{center}
% (此处填写摘要信息)
% \begin{enumerate}
%     \item 每个参赛队可以从A、B、C、D、E、F题中任选一题完成论文.
%     \item 论文题目和摘要写在论文摘要上,摘要页的下一页开始论文正文.
%     \item 论文从摘要页开始编写页码,页码必须位于每页页脚中部,用阿拉伯数字从“1 ”开始连续编号.
%     \item 论文不能有页眉,论文中不能有任何可能显示答题人身份的标志.
%     \item 论文题目用三号黑体字、一级标题用四号黑体字,并居中.论文中其他汉字一律采用小四号宋体字,行距用单倍行距.计算机结果和源程序需在规定时间内上传竞赛系统以备检查.
%     \item 请大家注意:摘要应该是一份简明扼要的详细摘要(包括关键词),请认真书写(注意篇幅一般不超过两页,且无需译成英文).全国评阅时对摘要和论文都会审阅.
%     \item 引用别人的成果或其他公开的资料(包括网上甚至在“博客”上查到的资料) 必须按照规定的参考文献的表述方式在正文引用处和参考文献中明确列出.正文引用处用方括号标示参考文献的编号,如\cite{ref1}\cite{ref2}等;引用书籍还必须指出页码.参考文献按正文中的引用次序列出,其中书籍的表述方式为:
%     [编号] 作者,书名,出版地:出版社,起止页码,出版年.
%     参考文献中期刊杂志论文的表述方式为:
%     [编号] 作者,论文名,杂志名,卷期号:起止页码,出版年.
%     参考文献中网上资源的表述方式为:
%     [编号] 作者,资源标题,网址,访问时间(年月日).
% \end{enumerate}


  
    
\vspace{1em}
\noindent \textbf{关键词:} 
