
%--------------------附录----------------------
\newpage

\appendix

\section{主程序源代码}

\subsection{\texttt{DirectedGraphVisualization.m}}\label{sec A 4}
这段代码主要用于生成一个有向图并进行可视化。
\begin{lstlisting}[style=matlab,basicstyle=\footnotesize\fontspec{Courier New},title={DirectedGraphVisualization.m}]  
    % 设置随机数生成器的种子以获得可重复的结果
    rng(3);
    
    % 定义节点数量
    numNodes = 5;
    
    % 随机生成边(每个节点都可以连接到其他节点,但不包括自己)
    s = repmat(1:numNodes, 1, numNodes);
    t = repmat(1:numNodes, numNodes, 1);
    s = s(:)';
    t = t(:)';
    
    % 删除自连接的边
    selfLoops = s == t;
    s(selfLoops) = [];
    t(selfLoops) = [];
    
    % 随机生成边的权重和节点的权重
    edgeWeights = round((0.01 + (1-0.01)*rand(1, length(s))) * 100) / 100;
    nodeWeights = randi([20, 50], 1, numNodes);
    
    % 使用边和权重创建有向图对象
    G = digraph(s,t,edgeWeights);
    
    % 绘制有向图,并明确不显示节点标签
    h = plot(G, 'EdgeLabel', G.Edges.Weight, 'LineWidth', 2, 'ArrowSize', 10, 'NodeFontSize', 12, 'NodeColor', Color_Nature(1,:), 'EdgeColor', Color_Nature(2,:), 'EdgeAlpha', 0.7, 'ArrowPosition', 0.6, 'NodeLabel', {});
    
    % 使用节点的权重来调整节点的大小
    h.MarkerSize = nodeWeights;
    
    % 获取节点的坐标
    x = h.XData;
    y = h.YData;
    
    % 添加编号到节点的中心
    labels = arrayfun(@(x) num2str(x), 1:numNodes, 'UniformOutput', false);
    for i = 1:numNodes
        text(x(i), y(i), labels{i}, ...
            'HorizontalAlignment', 'center', ...
            'VerticalAlignment', 'middle', ...
            'Color', Color_Nature(3,:), ...
            'FontSize', 14, ...  % 设置字体大小,您可以根据需要调整
            'FontWeight', 'bold');  % 设置为加粗
    end

\end{lstlisting}